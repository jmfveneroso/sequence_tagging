Web Data Extraction methods often rely on hand-coded rules to 
identify and extract data from webpages. These methods are
suited for extracting information from pages within
the same website, however they perform poorly on extraction 
tasks across different websites. Alternatively, statistical and 
machine-learning-based Named Entity Recognition (NER) methods provide a more flexible 
approach to Web Data Extraction. This is important, 
because sentences in HTML pages are often too short to provide adequate 
context for conventional NER methods to work 
properly. Nonetheless, the HTML structure also encodes useful information that 
can be used by NER models to achieve a better performance. We propose two
methods to use this information: the self-training strategy for Hidden Markov
Models and the hard attention mechanism for Bi-LSTM-CRFs, a type of neural network.
Also, in this dissertation we 
evaluate the performance of different methods of NER
in the task of Web Data Extraction. In particular, we introduce a novel 
dataset consisting of faculty listings from university webpages across
the world in multiple languages and test different NER models in the task of 
extracting researcher names from these listings. We found that a 
neural network architecture that combines a bidirectional LSTM with
a Conditional Random Fields output layer, LSTM-based character 
representations and a Hard Attention mechanism for HTML features
outperforms other methods achieving an F1-score of 90.7 in the task.
But, with the aid of clever strategies such as self-training, we can get a 
much simpler model, the second-order Hidden Markov Model, 
to achieve a 87.9 F1-score.


\keywords{Named Entity Recognition, Web Data Extraction, Researcher Name Extraction}
