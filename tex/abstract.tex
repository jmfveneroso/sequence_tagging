Web data extraction methods often rely on hand-coded rules to 
identify and extract data from webpages. These methods are usually
suited for extracting information from pages within
the same website, however they perform poorly on extraction 
tasks across different websites. Alternatively, statistical and 
machine-learning-based sequence labeling methods provide a more flexible 
approach to Web data extraction. Many times, HTML pages are very different 
from plain text, because sentences are too short to provide adequate 
context for conventional Named Entity Recognition methods to work 
properly. Also, the HTML structure may encode information that is not 
replicated in the text. Nonetheless, these limitations can be overcome by
adequate feature engineering, the use of pretrained word 
embeddings and neural character representations. In this article, we 
evaluate the performance of different methods of named entity recognition 
on the task of Web data extraction. In particular, we introduce a novel 
dataset\footnote{The dataset and all models discussed in this article are 
available in: https://github.com/jmfveneroso/ner-on-html.} 
consisting of faculty listings from university webpages across
the world in multiple languages and test the NER models on the task of 
extracting researcher names from these listings. We found that a 
neural network architecture that combines a bidirectional LSTM with
a Conditional Random Fields output layer and LSTM-based character 
representations outperforms other methods on the researcher name 
extraction task, achieving an F1-score of 0.8867 with no feature engineering. 
With the addition of hand crafted features, the F1-score can be slightly 
improved to 0.8995.

\keywords{Named entity recognition, information extraction, web data extraction}
