Web Data Extraction methods often rely on hand-coded rules to 
identify and extract data from webpages. These methods are
suited for extracting information from pages within
the same website, however they perform poorly on extraction 
tasks across different websites. Alternatively, statistical and 
machine-learning-based sequence labeling methods provide a more flexible 
approach to Web Data Extraction. Many times, HTML pages are very different 
from plain text, because sentences are too short to provide adequate 
context for conventional Named Entity Recognition methods to work 
properly. Also, the HTML structure may encode information that is not 
replicated in the text. Nonetheless, these limitations can be overcome by
adequate feature engineering and the Self-Training strategy for Hidden
Markov Models or with the use of pre-trained word embeddings, neural character 
representations and the hard attention mechanism in Bi-LSTM-CRFs.
In this dissertation, we 
evaluate the performance of different methods of Named Entity Recognition 
in the task of Web Data Extraction. In particular, we introduce a novel 
dataset consisting of faculty listings from university webpages across
the world in multiple languages and test different NER models in the task of 
extracting researcher names from these listings. We found that a 
neural network architecture that combines a bidirectional LSTM with
a Conditional Random Fields output layer, LSTM-based character 
representations and a Hard Attention mechanism for HTML features
outperforms other methods achieving an F1-score of 0.902 in the task.
But with some clever strategies, we can get a much simpler model, the 
second-order Hidden Markov Model with self-training, to achieve a 0.879 
F1-score.


\keywords{Named entity recognition, Information Extraction, Web Data Extraction}
